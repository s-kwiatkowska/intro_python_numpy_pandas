\documentclass[11pt]{article}

    \usepackage[breakable]{tcolorbox}
    \usepackage{parskip} % Stop auto-indenting (to mimic markdown behaviour)
    
    \usepackage{iftex}
    \ifPDFTeX
    	\usepackage[T1]{fontenc}
    	\usepackage{mathpazo}
    \else
    	\usepackage{fontspec}
    \fi

    % Basic figure setup, for now with no caption control since it's done
    % automatically by Pandoc (which extracts ![](path) syntax from Markdown).
    \usepackage{graphicx}
    % Maintain compatibility with old templates. Remove in nbconvert 6.0
    \let\Oldincludegraphics\includegraphics
    % Ensure that by default, figures have no caption (until we provide a
    % proper Figure object with a Caption API and a way to capture that
    % in the conversion process - todo).
    \usepackage{caption}
    \DeclareCaptionFormat{nocaption}{}
    \captionsetup{format=nocaption,aboveskip=0pt,belowskip=0pt}

    \usepackage[Export]{adjustbox} % Used to constrain images to a maximum size
    \adjustboxset{max size={0.9\linewidth}{0.9\paperheight}}
    \usepackage{float}
    \floatplacement{figure}{H} % forces figures to be placed at the correct location
    \usepackage{xcolor} % Allow colors to be defined
    \usepackage{enumerate} % Needed for markdown enumerations to work
    \usepackage{geometry} % Used to adjust the document margins
    \usepackage{amsmath} % Equations
    \usepackage{amssymb} % Equations
    \usepackage{textcomp} % defines textquotesingle
    % Hack from http://tex.stackexchange.com/a/47451/13684:
    \AtBeginDocument{%
        \def\PYZsq{\textquotesingle}% Upright quotes in Pygmentized code
    }
    \usepackage{upquote} % Upright quotes for verbatim code
    \usepackage{eurosym} % defines \euro
    \usepackage[mathletters]{ucs} % Extended unicode (utf-8) support
    \usepackage{fancyvrb} % verbatim replacement that allows latex
    \usepackage{grffile} % extends the file name processing of package graphics 
                         % to support a larger range
    \makeatletter % fix for grffile with XeLaTeX
    \def\Gread@@xetex#1{%
      \IfFileExists{"\Gin@base".bb}%
      {\Gread@eps{\Gin@base.bb}}%
      {\Gread@@xetex@aux#1}%
    }
    \makeatother

    % The hyperref package gives us a pdf with properly built
    % internal navigation ('pdf bookmarks' for the table of contents,
    % internal cross-reference links, web links for URLs, etc.)
    \usepackage{hyperref}
    % The default LaTeX title has an obnoxious amount of whitespace. By default,
    % titling removes some of it. It also provides customization options.
    \usepackage{titling}
    \usepackage{longtable} % longtable support required by pandoc >1.10
    \usepackage{booktabs}  % table support for pandoc > 1.12.2
    \usepackage[inline]{enumitem} % IRkernel/repr support (it uses the enumerate* environment)
    \usepackage[normalem]{ulem} % ulem is needed to support strikethroughs (\sout)
                                % normalem makes italics be italics, not underlines
    \usepackage{mathrsfs}
    

    
    % Colors for the hyperref package
    \definecolor{urlcolor}{rgb}{0,.145,.698}
    \definecolor{linkcolor}{rgb}{.71,0.21,0.01}
    \definecolor{citecolor}{rgb}{.12,.54,.11}

    % ANSI colors
    \definecolor{ansi-black}{HTML}{3E424D}
    \definecolor{ansi-black-intense}{HTML}{282C36}
    \definecolor{ansi-red}{HTML}{E75C58}
    \definecolor{ansi-red-intense}{HTML}{B22B31}
    \definecolor{ansi-green}{HTML}{00A250}
    \definecolor{ansi-green-intense}{HTML}{007427}
    \definecolor{ansi-yellow}{HTML}{DDB62B}
    \definecolor{ansi-yellow-intense}{HTML}{B27D12}
    \definecolor{ansi-blue}{HTML}{208FFB}
    \definecolor{ansi-blue-intense}{HTML}{0065CA}
    \definecolor{ansi-magenta}{HTML}{D160C4}
    \definecolor{ansi-magenta-intense}{HTML}{A03196}
    \definecolor{ansi-cyan}{HTML}{60C6C8}
    \definecolor{ansi-cyan-intense}{HTML}{258F8F}
    \definecolor{ansi-white}{HTML}{C5C1B4}
    \definecolor{ansi-white-intense}{HTML}{A1A6B2}
    \definecolor{ansi-default-inverse-fg}{HTML}{FFFFFF}
    \definecolor{ansi-default-inverse-bg}{HTML}{000000}

    % commands and environments needed by pandoc snippets
    % extracted from the output of `pandoc -s`
    \providecommand{\tightlist}{%
      \setlength{\itemsep}{0pt}\setlength{\parskip}{0pt}}
    \DefineVerbatimEnvironment{Highlighting}{Verbatim}{commandchars=\\\{\}}
    % Add ',fontsize=\small' for more characters per line
    \newenvironment{Shaded}{}{}
    \newcommand{\KeywordTok}[1]{\textcolor[rgb]{0.00,0.44,0.13}{\textbf{{#1}}}}
    \newcommand{\DataTypeTok}[1]{\textcolor[rgb]{0.56,0.13,0.00}{{#1}}}
    \newcommand{\DecValTok}[1]{\textcolor[rgb]{0.25,0.63,0.44}{{#1}}}
    \newcommand{\BaseNTok}[1]{\textcolor[rgb]{0.25,0.63,0.44}{{#1}}}
    \newcommand{\FloatTok}[1]{\textcolor[rgb]{0.25,0.63,0.44}{{#1}}}
    \newcommand{\CharTok}[1]{\textcolor[rgb]{0.25,0.44,0.63}{{#1}}}
    \newcommand{\StringTok}[1]{\textcolor[rgb]{0.25,0.44,0.63}{{#1}}}
    \newcommand{\CommentTok}[1]{\textcolor[rgb]{0.38,0.63,0.69}{\textit{{#1}}}}
    \newcommand{\OtherTok}[1]{\textcolor[rgb]{0.00,0.44,0.13}{{#1}}}
    \newcommand{\AlertTok}[1]{\textcolor[rgb]{1.00,0.00,0.00}{\textbf{{#1}}}}
    \newcommand{\FunctionTok}[1]{\textcolor[rgb]{0.02,0.16,0.49}{{#1}}}
    \newcommand{\RegionMarkerTok}[1]{{#1}}
    \newcommand{\ErrorTok}[1]{\textcolor[rgb]{1.00,0.00,0.00}{\textbf{{#1}}}}
    \newcommand{\NormalTok}[1]{{#1}}
    
    % Additional commands for more recent versions of Pandoc
    \newcommand{\ConstantTok}[1]{\textcolor[rgb]{0.53,0.00,0.00}{{#1}}}
    \newcommand{\SpecialCharTok}[1]{\textcolor[rgb]{0.25,0.44,0.63}{{#1}}}
    \newcommand{\VerbatimStringTok}[1]{\textcolor[rgb]{0.25,0.44,0.63}{{#1}}}
    \newcommand{\SpecialStringTok}[1]{\textcolor[rgb]{0.73,0.40,0.53}{{#1}}}
    \newcommand{\ImportTok}[1]{{#1}}
    \newcommand{\DocumentationTok}[1]{\textcolor[rgb]{0.73,0.13,0.13}{\textit{{#1}}}}
    \newcommand{\AnnotationTok}[1]{\textcolor[rgb]{0.38,0.63,0.69}{\textbf{\textit{{#1}}}}}
    \newcommand{\CommentVarTok}[1]{\textcolor[rgb]{0.38,0.63,0.69}{\textbf{\textit{{#1}}}}}
    \newcommand{\VariableTok}[1]{\textcolor[rgb]{0.10,0.09,0.49}{{#1}}}
    \newcommand{\ControlFlowTok}[1]{\textcolor[rgb]{0.00,0.44,0.13}{\textbf{{#1}}}}
    \newcommand{\OperatorTok}[1]{\textcolor[rgb]{0.40,0.40,0.40}{{#1}}}
    \newcommand{\BuiltInTok}[1]{{#1}}
    \newcommand{\ExtensionTok}[1]{{#1}}
    \newcommand{\PreprocessorTok}[1]{\textcolor[rgb]{0.74,0.48,0.00}{{#1}}}
    \newcommand{\AttributeTok}[1]{\textcolor[rgb]{0.49,0.56,0.16}{{#1}}}
    \newcommand{\InformationTok}[1]{\textcolor[rgb]{0.38,0.63,0.69}{\textbf{\textit{{#1}}}}}
    \newcommand{\WarningTok}[1]{\textcolor[rgb]{0.38,0.63,0.69}{\textbf{\textit{{#1}}}}}
    
    
    % Define a nice break command that doesn't care if a line doesn't already
    % exist.
    \def\br{\hspace*{\fill} \\* }
    % Math Jax compatibility definitions
    \def\gt{>}
    \def\lt{<}
    \let\Oldtex\TeX
    \let\Oldlatex\LaTeX
    \renewcommand{\TeX}{\textrm{\Oldtex}}
    \renewcommand{\LaTeX}{\textrm{\Oldlatex}}
    % Document parameters
    % Document title
    \title{lesson1.2\_intro\_python\_numpy\_pandas}
    
    
    
    
    
% Pygments definitions
\makeatletter
\def\PY@reset{\let\PY@it=\relax \let\PY@bf=\relax%
    \let\PY@ul=\relax \let\PY@tc=\relax%
    \let\PY@bc=\relax \let\PY@ff=\relax}
\def\PY@tok#1{\csname PY@tok@#1\endcsname}
\def\PY@toks#1+{\ifx\relax#1\empty\else%
    \PY@tok{#1}\expandafter\PY@toks\fi}
\def\PY@do#1{\PY@bc{\PY@tc{\PY@ul{%
    \PY@it{\PY@bf{\PY@ff{#1}}}}}}}
\def\PY#1#2{\PY@reset\PY@toks#1+\relax+\PY@do{#2}}

\expandafter\def\csname PY@tok@w\endcsname{\def\PY@tc##1{\textcolor[rgb]{0.73,0.73,0.73}{##1}}}
\expandafter\def\csname PY@tok@c\endcsname{\let\PY@it=\textit\def\PY@tc##1{\textcolor[rgb]{0.25,0.50,0.50}{##1}}}
\expandafter\def\csname PY@tok@cp\endcsname{\def\PY@tc##1{\textcolor[rgb]{0.74,0.48,0.00}{##1}}}
\expandafter\def\csname PY@tok@k\endcsname{\let\PY@bf=\textbf\def\PY@tc##1{\textcolor[rgb]{0.00,0.50,0.00}{##1}}}
\expandafter\def\csname PY@tok@kp\endcsname{\def\PY@tc##1{\textcolor[rgb]{0.00,0.50,0.00}{##1}}}
\expandafter\def\csname PY@tok@kt\endcsname{\def\PY@tc##1{\textcolor[rgb]{0.69,0.00,0.25}{##1}}}
\expandafter\def\csname PY@tok@o\endcsname{\def\PY@tc##1{\textcolor[rgb]{0.40,0.40,0.40}{##1}}}
\expandafter\def\csname PY@tok@ow\endcsname{\let\PY@bf=\textbf\def\PY@tc##1{\textcolor[rgb]{0.67,0.13,1.00}{##1}}}
\expandafter\def\csname PY@tok@nb\endcsname{\def\PY@tc##1{\textcolor[rgb]{0.00,0.50,0.00}{##1}}}
\expandafter\def\csname PY@tok@nf\endcsname{\def\PY@tc##1{\textcolor[rgb]{0.00,0.00,1.00}{##1}}}
\expandafter\def\csname PY@tok@nc\endcsname{\let\PY@bf=\textbf\def\PY@tc##1{\textcolor[rgb]{0.00,0.00,1.00}{##1}}}
\expandafter\def\csname PY@tok@nn\endcsname{\let\PY@bf=\textbf\def\PY@tc##1{\textcolor[rgb]{0.00,0.00,1.00}{##1}}}
\expandafter\def\csname PY@tok@ne\endcsname{\let\PY@bf=\textbf\def\PY@tc##1{\textcolor[rgb]{0.82,0.25,0.23}{##1}}}
\expandafter\def\csname PY@tok@nv\endcsname{\def\PY@tc##1{\textcolor[rgb]{0.10,0.09,0.49}{##1}}}
\expandafter\def\csname PY@tok@no\endcsname{\def\PY@tc##1{\textcolor[rgb]{0.53,0.00,0.00}{##1}}}
\expandafter\def\csname PY@tok@nl\endcsname{\def\PY@tc##1{\textcolor[rgb]{0.63,0.63,0.00}{##1}}}
\expandafter\def\csname PY@tok@ni\endcsname{\let\PY@bf=\textbf\def\PY@tc##1{\textcolor[rgb]{0.60,0.60,0.60}{##1}}}
\expandafter\def\csname PY@tok@na\endcsname{\def\PY@tc##1{\textcolor[rgb]{0.49,0.56,0.16}{##1}}}
\expandafter\def\csname PY@tok@nt\endcsname{\let\PY@bf=\textbf\def\PY@tc##1{\textcolor[rgb]{0.00,0.50,0.00}{##1}}}
\expandafter\def\csname PY@tok@nd\endcsname{\def\PY@tc##1{\textcolor[rgb]{0.67,0.13,1.00}{##1}}}
\expandafter\def\csname PY@tok@s\endcsname{\def\PY@tc##1{\textcolor[rgb]{0.73,0.13,0.13}{##1}}}
\expandafter\def\csname PY@tok@sd\endcsname{\let\PY@it=\textit\def\PY@tc##1{\textcolor[rgb]{0.73,0.13,0.13}{##1}}}
\expandafter\def\csname PY@tok@si\endcsname{\let\PY@bf=\textbf\def\PY@tc##1{\textcolor[rgb]{0.73,0.40,0.53}{##1}}}
\expandafter\def\csname PY@tok@se\endcsname{\let\PY@bf=\textbf\def\PY@tc##1{\textcolor[rgb]{0.73,0.40,0.13}{##1}}}
\expandafter\def\csname PY@tok@sr\endcsname{\def\PY@tc##1{\textcolor[rgb]{0.73,0.40,0.53}{##1}}}
\expandafter\def\csname PY@tok@ss\endcsname{\def\PY@tc##1{\textcolor[rgb]{0.10,0.09,0.49}{##1}}}
\expandafter\def\csname PY@tok@sx\endcsname{\def\PY@tc##1{\textcolor[rgb]{0.00,0.50,0.00}{##1}}}
\expandafter\def\csname PY@tok@m\endcsname{\def\PY@tc##1{\textcolor[rgb]{0.40,0.40,0.40}{##1}}}
\expandafter\def\csname PY@tok@gh\endcsname{\let\PY@bf=\textbf\def\PY@tc##1{\textcolor[rgb]{0.00,0.00,0.50}{##1}}}
\expandafter\def\csname PY@tok@gu\endcsname{\let\PY@bf=\textbf\def\PY@tc##1{\textcolor[rgb]{0.50,0.00,0.50}{##1}}}
\expandafter\def\csname PY@tok@gd\endcsname{\def\PY@tc##1{\textcolor[rgb]{0.63,0.00,0.00}{##1}}}
\expandafter\def\csname PY@tok@gi\endcsname{\def\PY@tc##1{\textcolor[rgb]{0.00,0.63,0.00}{##1}}}
\expandafter\def\csname PY@tok@gr\endcsname{\def\PY@tc##1{\textcolor[rgb]{1.00,0.00,0.00}{##1}}}
\expandafter\def\csname PY@tok@ge\endcsname{\let\PY@it=\textit}
\expandafter\def\csname PY@tok@gs\endcsname{\let\PY@bf=\textbf}
\expandafter\def\csname PY@tok@gp\endcsname{\let\PY@bf=\textbf\def\PY@tc##1{\textcolor[rgb]{0.00,0.00,0.50}{##1}}}
\expandafter\def\csname PY@tok@go\endcsname{\def\PY@tc##1{\textcolor[rgb]{0.53,0.53,0.53}{##1}}}
\expandafter\def\csname PY@tok@gt\endcsname{\def\PY@tc##1{\textcolor[rgb]{0.00,0.27,0.87}{##1}}}
\expandafter\def\csname PY@tok@err\endcsname{\def\PY@bc##1{\setlength{\fboxsep}{0pt}\fcolorbox[rgb]{1.00,0.00,0.00}{1,1,1}{\strut ##1}}}
\expandafter\def\csname PY@tok@kc\endcsname{\let\PY@bf=\textbf\def\PY@tc##1{\textcolor[rgb]{0.00,0.50,0.00}{##1}}}
\expandafter\def\csname PY@tok@kd\endcsname{\let\PY@bf=\textbf\def\PY@tc##1{\textcolor[rgb]{0.00,0.50,0.00}{##1}}}
\expandafter\def\csname PY@tok@kn\endcsname{\let\PY@bf=\textbf\def\PY@tc##1{\textcolor[rgb]{0.00,0.50,0.00}{##1}}}
\expandafter\def\csname PY@tok@kr\endcsname{\let\PY@bf=\textbf\def\PY@tc##1{\textcolor[rgb]{0.00,0.50,0.00}{##1}}}
\expandafter\def\csname PY@tok@bp\endcsname{\def\PY@tc##1{\textcolor[rgb]{0.00,0.50,0.00}{##1}}}
\expandafter\def\csname PY@tok@fm\endcsname{\def\PY@tc##1{\textcolor[rgb]{0.00,0.00,1.00}{##1}}}
\expandafter\def\csname PY@tok@vc\endcsname{\def\PY@tc##1{\textcolor[rgb]{0.10,0.09,0.49}{##1}}}
\expandafter\def\csname PY@tok@vg\endcsname{\def\PY@tc##1{\textcolor[rgb]{0.10,0.09,0.49}{##1}}}
\expandafter\def\csname PY@tok@vi\endcsname{\def\PY@tc##1{\textcolor[rgb]{0.10,0.09,0.49}{##1}}}
\expandafter\def\csname PY@tok@vm\endcsname{\def\PY@tc##1{\textcolor[rgb]{0.10,0.09,0.49}{##1}}}
\expandafter\def\csname PY@tok@sa\endcsname{\def\PY@tc##1{\textcolor[rgb]{0.73,0.13,0.13}{##1}}}
\expandafter\def\csname PY@tok@sb\endcsname{\def\PY@tc##1{\textcolor[rgb]{0.73,0.13,0.13}{##1}}}
\expandafter\def\csname PY@tok@sc\endcsname{\def\PY@tc##1{\textcolor[rgb]{0.73,0.13,0.13}{##1}}}
\expandafter\def\csname PY@tok@dl\endcsname{\def\PY@tc##1{\textcolor[rgb]{0.73,0.13,0.13}{##1}}}
\expandafter\def\csname PY@tok@s2\endcsname{\def\PY@tc##1{\textcolor[rgb]{0.73,0.13,0.13}{##1}}}
\expandafter\def\csname PY@tok@sh\endcsname{\def\PY@tc##1{\textcolor[rgb]{0.73,0.13,0.13}{##1}}}
\expandafter\def\csname PY@tok@s1\endcsname{\def\PY@tc##1{\textcolor[rgb]{0.73,0.13,0.13}{##1}}}
\expandafter\def\csname PY@tok@mb\endcsname{\def\PY@tc##1{\textcolor[rgb]{0.40,0.40,0.40}{##1}}}
\expandafter\def\csname PY@tok@mf\endcsname{\def\PY@tc##1{\textcolor[rgb]{0.40,0.40,0.40}{##1}}}
\expandafter\def\csname PY@tok@mh\endcsname{\def\PY@tc##1{\textcolor[rgb]{0.40,0.40,0.40}{##1}}}
\expandafter\def\csname PY@tok@mi\endcsname{\def\PY@tc##1{\textcolor[rgb]{0.40,0.40,0.40}{##1}}}
\expandafter\def\csname PY@tok@il\endcsname{\def\PY@tc##1{\textcolor[rgb]{0.40,0.40,0.40}{##1}}}
\expandafter\def\csname PY@tok@mo\endcsname{\def\PY@tc##1{\textcolor[rgb]{0.40,0.40,0.40}{##1}}}
\expandafter\def\csname PY@tok@ch\endcsname{\let\PY@it=\textit\def\PY@tc##1{\textcolor[rgb]{0.25,0.50,0.50}{##1}}}
\expandafter\def\csname PY@tok@cm\endcsname{\let\PY@it=\textit\def\PY@tc##1{\textcolor[rgb]{0.25,0.50,0.50}{##1}}}
\expandafter\def\csname PY@tok@cpf\endcsname{\let\PY@it=\textit\def\PY@tc##1{\textcolor[rgb]{0.25,0.50,0.50}{##1}}}
\expandafter\def\csname PY@tok@c1\endcsname{\let\PY@it=\textit\def\PY@tc##1{\textcolor[rgb]{0.25,0.50,0.50}{##1}}}
\expandafter\def\csname PY@tok@cs\endcsname{\let\PY@it=\textit\def\PY@tc##1{\textcolor[rgb]{0.25,0.50,0.50}{##1}}}

\def\PYZbs{\char`\\}
\def\PYZus{\char`\_}
\def\PYZob{\char`\{}
\def\PYZcb{\char`\}}
\def\PYZca{\char`\^}
\def\PYZam{\char`\&}
\def\PYZlt{\char`\<}
\def\PYZgt{\char`\>}
\def\PYZsh{\char`\#}
\def\PYZpc{\char`\%}
\def\PYZdl{\char`\$}
\def\PYZhy{\char`\-}
\def\PYZsq{\char`\'}
\def\PYZdq{\char`\"}
\def\PYZti{\char`\~}
% for compatibility with earlier versions
\def\PYZat{@}
\def\PYZlb{[}
\def\PYZrb{]}
\makeatother


    % For linebreaks inside Verbatim environment from package fancyvrb. 
    \makeatletter
        \newbox\Wrappedcontinuationbox 
        \newbox\Wrappedvisiblespacebox 
        \newcommand*\Wrappedvisiblespace {\textcolor{red}{\textvisiblespace}} 
        \newcommand*\Wrappedcontinuationsymbol {\textcolor{red}{\llap{\tiny$\m@th\hookrightarrow$}}} 
        \newcommand*\Wrappedcontinuationindent {3ex } 
        \newcommand*\Wrappedafterbreak {\kern\Wrappedcontinuationindent\copy\Wrappedcontinuationbox} 
        % Take advantage of the already applied Pygments mark-up to insert 
        % potential linebreaks for TeX processing. 
        %        {, <, #, %, $, ' and ": go to next line. 
        %        _, }, ^, &, >, - and ~: stay at end of broken line. 
        % Use of \textquotesingle for straight quote. 
        \newcommand*\Wrappedbreaksatspecials {% 
            \def\PYGZus{\discretionary{\char`\_}{\Wrappedafterbreak}{\char`\_}}% 
            \def\PYGZob{\discretionary{}{\Wrappedafterbreak\char`\{}{\char`\{}}% 
            \def\PYGZcb{\discretionary{\char`\}}{\Wrappedafterbreak}{\char`\}}}% 
            \def\PYGZca{\discretionary{\char`\^}{\Wrappedafterbreak}{\char`\^}}% 
            \def\PYGZam{\discretionary{\char`\&}{\Wrappedafterbreak}{\char`\&}}% 
            \def\PYGZlt{\discretionary{}{\Wrappedafterbreak\char`\<}{\char`\<}}% 
            \def\PYGZgt{\discretionary{\char`\>}{\Wrappedafterbreak}{\char`\>}}% 
            \def\PYGZsh{\discretionary{}{\Wrappedafterbreak\char`\#}{\char`\#}}% 
            \def\PYGZpc{\discretionary{}{\Wrappedafterbreak\char`\%}{\char`\%}}% 
            \def\PYGZdl{\discretionary{}{\Wrappedafterbreak\char`\$}{\char`\$}}% 
            \def\PYGZhy{\discretionary{\char`\-}{\Wrappedafterbreak}{\char`\-}}% 
            \def\PYGZsq{\discretionary{}{\Wrappedafterbreak\textquotesingle}{\textquotesingle}}% 
            \def\PYGZdq{\discretionary{}{\Wrappedafterbreak\char`\"}{\char`\"}}% 
            \def\PYGZti{\discretionary{\char`\~}{\Wrappedafterbreak}{\char`\~}}% 
        } 
        % Some characters . , ; ? ! / are not pygmentized. 
        % This macro makes them "active" and they will insert potential linebreaks 
        \newcommand*\Wrappedbreaksatpunct {% 
            \lccode`\~`\.\lowercase{\def~}{\discretionary{\hbox{\char`\.}}{\Wrappedafterbreak}{\hbox{\char`\.}}}% 
            \lccode`\~`\,\lowercase{\def~}{\discretionary{\hbox{\char`\,}}{\Wrappedafterbreak}{\hbox{\char`\,}}}% 
            \lccode`\~`\;\lowercase{\def~}{\discretionary{\hbox{\char`\;}}{\Wrappedafterbreak}{\hbox{\char`\;}}}% 
            \lccode`\~`\:\lowercase{\def~}{\discretionary{\hbox{\char`\:}}{\Wrappedafterbreak}{\hbox{\char`\:}}}% 
            \lccode`\~`\?\lowercase{\def~}{\discretionary{\hbox{\char`\?}}{\Wrappedafterbreak}{\hbox{\char`\?}}}% 
            \lccode`\~`\!\lowercase{\def~}{\discretionary{\hbox{\char`\!}}{\Wrappedafterbreak}{\hbox{\char`\!}}}% 
            \lccode`\~`\/\lowercase{\def~}{\discretionary{\hbox{\char`\/}}{\Wrappedafterbreak}{\hbox{\char`\/}}}% 
            \catcode`\.\active
            \catcode`\,\active 
            \catcode`\;\active
            \catcode`\:\active
            \catcode`\?\active
            \catcode`\!\active
            \catcode`\/\active 
            \lccode`\~`\~ 	
        }
    \makeatother

    \let\OriginalVerbatim=\Verbatim
    \makeatletter
    \renewcommand{\Verbatim}[1][1]{%
        %\parskip\z@skip
        \sbox\Wrappedcontinuationbox {\Wrappedcontinuationsymbol}%
        \sbox\Wrappedvisiblespacebox {\FV@SetupFont\Wrappedvisiblespace}%
        \def\FancyVerbFormatLine ##1{\hsize\linewidth
            \vtop{\raggedright\hyphenpenalty\z@\exhyphenpenalty\z@
                \doublehyphendemerits\z@\finalhyphendemerits\z@
                \strut ##1\strut}%
        }%
        % If the linebreak is at a space, the latter will be displayed as visible
        % space at end of first line, and a continuation symbol starts next line.
        % Stretch/shrink are however usually zero for typewriter font.
        \def\FV@Space {%
            \nobreak\hskip\z@ plus\fontdimen3\font minus\fontdimen4\font
            \discretionary{\copy\Wrappedvisiblespacebox}{\Wrappedafterbreak}
            {\kern\fontdimen2\font}%
        }%
        
        % Allow breaks at special characters using \PYG... macros.
        \Wrappedbreaksatspecials
        % Breaks at punctuation characters . , ; ? ! and / need catcode=\active 	
        \OriginalVerbatim[#1,codes*=\Wrappedbreaksatpunct]%
    }
    \makeatother

    % Exact colors from NB
    \definecolor{incolor}{HTML}{303F9F}
    \definecolor{outcolor}{HTML}{D84315}
    \definecolor{cellborder}{HTML}{CFCFCF}
    \definecolor{cellbackground}{HTML}{F7F7F7}
    
    % prompt
    \makeatletter
    \newcommand{\boxspacing}{\kern\kvtcb@left@rule\kern\kvtcb@boxsep}
    \makeatother
    \newcommand{\prompt}[4]{
        \ttfamily\llap{{\color{#2}[#3]:\hspace{3pt}#4}}\vspace{-\baselineskip}
    }
    

    
    % Prevent overflowing lines due to hard-to-break entities
    \sloppy 
    % Setup hyperref package
    \hypersetup{
      breaklinks=true,  % so long urls are correctly broken across lines
      colorlinks=true,
      urlcolor=urlcolor,
      linkcolor=linkcolor,
      citecolor=citecolor,
      }
    % Slightly bigger margins than the latex defaults
    
    \geometry{verbose,tmargin=1in,bmargin=1in,lmargin=1in,rmargin=1in}
    
    

\begin{document}
    
    \maketitle
    
    

    
    \hypertarget{python-i-biblioteki}{%
\section{Python i biblioteki}\label{python-i-biblioteki}}

\hypertarget{celem-jest-przypomnieux107-lub-poznaux107-skux142adniux119-pythona-oraz-potrzebnych-bibliotek.}{%
\subsubsection{Celem jest przypomnieć lub poznać składnię Pythona oraz
potrzebnych
bibliotek.}\label{celem-jest-przypomnieux107-lub-poznaux107-skux142adniux119-pythona-oraz-potrzebnych-bibliotek.}}

Te ćwiczenia są obowiązkowe, jeśli chcesz sobie przypomnieć lub poznać
potrzebne podstawy \href{https://bit.ly/2O0HJFQ}{Python} oraz potrzebne
biblioteki.

Oczywiście to będzie tylko kawałek wiedzy, która nam się przyda. Sam
język Python, jak i biblioteki, wymagają znacznie więcej czasu niż
godzinka, ale musimy od czegoś zacząć :).

\textbf{Uwaga!} Jeśli czujesz się pewnie z Python i bibliotekami:
\href{https://bit.ly/3sAqEl2}{numpy},
\href{https://bit.ly/2O4s8Fm}{pandas} to zapraszam do kolejnego
ćwiczenia -
\textbf{\href{./lesson1.3_basic_model_predict_name.ipynb}{lesson1.3}}.

Natomiast jeśli czujesz, że jest to zbyt trudne, to przypominam o
notebooku ``\href{step_by_step.ipynb}{krok po kroku}'', w którym na
wideo jeszcze dokładniej tłumaczę, co się dzieje.

    \hypertarget{krok-po-kroku}{%
\subsubsection{Krok po kroku}\label{krok-po-kroku}}

Jeśli wolisz najpierw słuchać i oglądać, to obejrzyj nagranie poniżej,
które omawia tę lekcję.

    \begin{tcolorbox}[breakable, size=fbox, boxrule=1pt, pad at break*=1mm,colback=cellbackground, colframe=cellborder]
\prompt{In}{incolor}{1}{\boxspacing}
\begin{Verbatim}[commandchars=\\\{\}]
\PY{o}{\PYZpc{}\PYZpc{}html}
\PY{p}{\PYZlt{}}\PY{n+nt}{iframe} \PY{n+na}{style}\PY{o}{=}\PY{l+s}{\PYZdq{}height:500px;width:100\PYZpc{}\PYZdq{}} \PY{n+na}{src}\PY{o}{=}\PY{l+s}{\PYZdq{}https://bit.ly/39pJ40n\PYZdq{}} \PY{n+na}{frameborder}\PY{o}{=}\PY{l+s}{\PYZdq{}0\PYZdq{}} \PY{n+na}{allow}\PY{o}{=}\PY{l+s}{\PYZdq{}autoplay; encrypted\PYZhy{}media\PYZdq{}} \PY{n+na}{allowfullscreen}\PY{p}{\PYZgt{}}\PY{p}{\PYZlt{}}\PY{p}{/}\PY{n+nt}{iframe}\PY{p}{\PYZgt{}}
\end{Verbatim}
\end{tcolorbox}

    
    \begin{verbatim}
<IPython.core.display.HTML object>
    \end{verbatim}

    
    \hypertarget{python-3.7}{%
\subsection{Python (3.7)}\label{python-3.7}}

    \begin{tcolorbox}[breakable, size=fbox, boxrule=1pt, pad at break*=1mm,colback=cellbackground, colframe=cellborder]
\prompt{In}{incolor}{1}{\boxspacing}
\begin{Verbatim}[commandchars=\\\{\}]
\PY{n+nb}{print}\PY{p}{(}\PY{l+s+s2}{\PYZdq{}}\PY{l+s+s2}{Hello world!}\PY{l+s+s2}{\PYZdq{}}\PY{p}{)}
\end{Verbatim}
\end{tcolorbox}

    \begin{Verbatim}[commandchars=\\\{\}]
Hello world!
    \end{Verbatim}

    \textbf{Uwaga!} Wykonując jakąś komórkę w \emph{Jupyter}, zwracana jest
wartość ostatniego wiersza. W następnej komórce jest zakomentowany
wiersz z \texttt{\#print(name)}. Pamiętaj jednak, że samo
\texttt{"name"} da dokładnie to, co w cudzysłowiu. Możesz to sprawdzić.

    \begin{tcolorbox}[breakable, size=fbox, boxrule=1pt, pad at break*=1mm,colback=cellbackground, colframe=cellborder]
\prompt{In}{incolor}{2}{\boxspacing}
\begin{Verbatim}[commandchars=\\\{\}]
\PY{n}{name} \PY{o}{=} \PY{l+s+s2}{\PYZdq{}}\PY{l+s+s2}{Jacek}\PY{l+s+s2}{\PYZdq{}}
\PY{n}{name}
\PY{c+c1}{\PYZsh{}print(name)}
\end{Verbatim}
\end{tcolorbox}

            \begin{tcolorbox}[breakable, size=fbox, boxrule=.5pt, pad at break*=1mm, opacityfill=0]
\prompt{Out}{outcolor}{2}{\boxspacing}
\begin{Verbatim}[commandchars=\\\{\}]
'Jacek'
\end{Verbatim}
\end{tcolorbox}
        
    \begin{tcolorbox}[breakable, size=fbox, boxrule=1pt, pad at break*=1mm,colback=cellbackground, colframe=cellborder]
\prompt{In}{incolor}{1}{\boxspacing}
\begin{Verbatim}[commandchars=\\\{\}]
\PY{n}{name} \PY{o}{=} \PY{l+s+s2}{\PYZdq{}}\PY{l+s+s2}{Sandra}\PY{l+s+s2}{\PYZdq{}}
\PY{n}{name}
\end{Verbatim}
\end{tcolorbox}

            \begin{tcolorbox}[breakable, size=fbox, boxrule=.5pt, pad at break*=1mm, opacityfill=0]
\prompt{Out}{outcolor}{1}{\boxspacing}
\begin{Verbatim}[commandchars=\\\{\}]
'Sandra'
\end{Verbatim}
\end{tcolorbox}
        
    \begin{tcolorbox}[breakable, size=fbox, boxrule=1pt, pad at break*=1mm,colback=cellbackground, colframe=cellborder]
\prompt{In}{incolor}{3}{\boxspacing}
\begin{Verbatim}[commandchars=\\\{\}]
\PY{n}{years} \PY{o}{=} \PY{p}{[}\PY{l+m+mi}{2015}\PY{p}{,} \PY{l+m+mi}{2016}\PY{p}{,} \PY{l+m+mi}{2017}\PY{p}{,} \PY{l+m+mi}{2018}\PY{p}{]}
\PY{c+c1}{\PYZsh{}print(years)}
\PY{n}{years}
\end{Verbatim}
\end{tcolorbox}

            \begin{tcolorbox}[breakable, size=fbox, boxrule=.5pt, pad at break*=1mm, opacityfill=0]
\prompt{Out}{outcolor}{3}{\boxspacing}
\begin{Verbatim}[commandchars=\\\{\}]
[2015, 2016, 2017, 2018]
\end{Verbatim}
\end{tcolorbox}
        
    \begin{tcolorbox}[breakable, size=fbox, boxrule=1pt, pad at break*=1mm,colback=cellbackground, colframe=cellborder]
\prompt{In}{incolor}{3}{\boxspacing}
\begin{Verbatim}[commandchars=\\\{\}]
\PY{n}{years} \PY{o}{=} \PY{p}{[}\PY{l+m+mi}{2020}\PY{p}{,} \PY{l+m+mi}{2021}\PY{p}{,} \PY{l+m+mi}{2022}\PY{p}{,} \PY{l+m+mi}{2023}\PY{p}{]}
\PY{n+nb}{print}\PY{p}{(}\PY{n}{years}\PY{p}{)}
\end{Verbatim}
\end{tcolorbox}

    \begin{Verbatim}[commandchars=\\\{\}]
[2020, 2021, 2022, 2023]
    \end{Verbatim}

    \begin{tcolorbox}[breakable, size=fbox, boxrule=1pt, pad at break*=1mm,colback=cellbackground, colframe=cellborder]
\prompt{In}{incolor}{4}{\boxspacing}
\begin{Verbatim}[commandchars=\\\{\}]
\PY{n}{years} \PY{o}{=} \PY{n+nb}{list}\PY{p}{(}\PY{n+nb}{range}\PY{p}{(}\PY{l+m+mi}{2015}\PY{p}{,} \PY{l+m+mi}{2019}\PY{p}{)}\PY{p}{)} \PY{c+c1}{\PYZsh{}range(start, finish + 1)}
\PY{n}{years}
\end{Verbatim}
\end{tcolorbox}

            \begin{tcolorbox}[breakable, size=fbox, boxrule=.5pt, pad at break*=1mm, opacityfill=0]
\prompt{Out}{outcolor}{4}{\boxspacing}
\begin{Verbatim}[commandchars=\\\{\}]
[2015, 2016, 2017, 2018]
\end{Verbatim}
\end{tcolorbox}
        
    \textbf{Uwaga!}

Funkcja \texttt{range} generuje przedział od \ldots{} do, natomiast
zwróć uwagę, że \texttt{range} zatrzymuje się tuż przed ``do'' (nie
włączając). Na przykład jeśli chcemy wygenerować przedział do roku 2018,
to należy podać o jeden więcej (czyli 2019). Wtedy \texttt{range}
zatrzyma się tuż przed (czyli na roku 2018).

Stąd jest w komentarzu: \texttt{finish\ +\ 1}, do oczekiwanego końca
należy dodać jeszcze jeden.

    \begin{tcolorbox}[breakable, size=fbox, boxrule=1pt, pad at break*=1mm,colback=cellbackground, colframe=cellborder]
\prompt{In}{incolor}{5}{\boxspacing}
\begin{Verbatim}[commandchars=\\\{\}]
\PY{n+nb}{print}\PY{p}{(}\PY{n}{years}\PY{p}{[}\PY{l+m+mi}{0}\PY{p}{]}\PY{p}{)} \PY{c+c1}{\PYZsh{}pierwszy element}
\PY{n+nb}{print}\PY{p}{(}\PY{n}{years}\PY{p}{[}\PY{l+m+mi}{1}\PY{p}{]}\PY{p}{)} \PY{c+c1}{\PYZsh{}drugi element}
\PY{n+nb}{print}\PY{p}{(}\PY{n}{years}\PY{p}{[}\PY{o}{\PYZhy{}}\PY{l+m+mi}{1}\PY{p}{]}\PY{p}{)} \PY{c+c1}{\PYZsh{}ostatni element}
\PY{n+nb}{print}\PY{p}{(}\PY{n}{years}\PY{p}{[}\PY{o}{\PYZhy{}}\PY{l+m+mi}{2}\PY{p}{]}\PY{p}{)} \PY{c+c1}{\PYZsh{}przedostatni element}
\end{Verbatim}
\end{tcolorbox}

    \begin{Verbatim}[commandchars=\\\{\}]
2015
2016
2018
2017
    \end{Verbatim}

    \begin{tcolorbox}[breakable, size=fbox, boxrule=1pt, pad at break*=1mm,colback=cellbackground, colframe=cellborder]
\prompt{In}{incolor}{6}{\boxspacing}
\begin{Verbatim}[commandchars=\\\{\}]
\PY{n}{name} \PY{o}{=} \PY{l+s+s2}{\PYZdq{}}\PY{l+s+s2}{Jan Kowalski}\PY{l+s+s2}{\PYZdq{}}
\PY{n+nb}{print}\PY{p}{(}\PY{n}{name}\PY{p}{[}\PY{l+m+mi}{0}\PY{p}{]}\PY{p}{)} \PY{c+c1}{\PYZsh{}pierwsza litera}
\PY{n+nb}{print}\PY{p}{(}\PY{n}{name}\PY{p}{[}\PY{o}{\PYZhy{}}\PY{l+m+mi}{1}\PY{p}{]}\PY{p}{)} \PY{c+c1}{\PYZsh{}ostatnia litera}
\PY{n+nb}{print}\PY{p}{(}\PY{n}{name}\PY{o}{.}\PY{n}{lower}\PY{p}{(}\PY{p}{)}\PY{p}{)}\PY{c+c1}{\PYZsh{}małe litery}
\PY{n+nb}{print}\PY{p}{(}\PY{n}{name}\PY{o}{.}\PY{n}{upper}\PY{p}{(}\PY{p}{)}\PY{p}{)}\PY{c+c1}{\PYZsh{}duże litery}
\PY{n+nb}{print}\PY{p}{(}\PY{n}{name}\PY{o}{.}\PY{n}{split}\PY{p}{(}\PY{l+s+s1}{\PYZsq{}}\PY{l+s+s1}{ }\PY{l+s+s1}{\PYZsq{}}\PY{p}{)}\PY{p}{)} \PY{c+c1}{\PYZsh{}rozdziel string na listę używając danego separatora (w tym przypadku jest to spacja)}
\end{Verbatim}
\end{tcolorbox}

    \begin{Verbatim}[commandchars=\\\{\}]
J
i
jan kowalski
JAN KOWALSKI
['Jan', 'Kowalski']
    \end{Verbatim}

    \begin{tcolorbox}[breakable, size=fbox, boxrule=1pt, pad at break*=1mm,colback=cellbackground, colframe=cellborder]
\prompt{In}{incolor}{7}{\boxspacing}
\begin{Verbatim}[commandchars=\\\{\}]
\PY{n}{years} \PY{o}{=} \PY{n+nb}{range}\PY{p}{(}\PY{l+m+mi}{2010}\PY{p}{,} \PY{l+m+mi}{2019}\PY{p}{)}
\PY{n+nb}{print}\PY{p}{(}\PY{p}{[}\PY{n}{year} \PY{k}{for} \PY{n}{year} \PY{o+ow}{in} \PY{n}{years}\PY{p}{]}\PY{p}{)}
\PY{n+nb}{print}\PY{p}{(}\PY{p}{[}\PY{n}{year} \PY{o}{*} \PY{l+m+mi}{2} \PY{k}{for} \PY{n}{year} \PY{o+ow}{in} \PY{n}{years}\PY{p}{]}\PY{p}{)} \PY{c+c1}{\PYZsh{}podwojone wartości}
\PY{n+nb}{print}\PY{p}{(}\PY{p}{[}\PY{n}{year} \PY{o}{*} \PY{l+m+mi}{2} \PY{k}{for} \PY{n}{year} \PY{o+ow}{in} \PY{n}{years} \PY{k}{if} \PY{n}{year} \PY{o}{\PYZpc{}} \PY{l+m+mi}{2} \PY{o}{==} \PY{l+m+mi}{0}\PY{p}{]}\PY{p}{)} \PY{c+c1}{\PYZsh{}podwój wartość, jeśli to liczba parzysta}
\end{Verbatim}
\end{tcolorbox}

    \begin{Verbatim}[commandchars=\\\{\}]
[2010, 2011, 2012, 2013, 2014, 2015, 2016, 2017, 2018]
[4020, 4022, 4024, 4026, 4028, 4030, 4032, 4034, 4036]
[4020, 4024, 4028, 4032, 4036]
    \end{Verbatim}

    \begin{tcolorbox}[breakable, size=fbox, boxrule=1pt, pad at break*=1mm,colback=cellbackground, colframe=cellborder]
\prompt{In}{incolor}{8}{\boxspacing}
\begin{Verbatim}[commandchars=\\\{\}]
\PY{p}{[}\PY{n}{x}\PY{o}{\PYZhy{}}\PY{l+m+mi}{50} \PY{k}{for} \PY{n}{x} \PY{o+ow}{in} \PY{n}{years}\PY{p}{]}
\end{Verbatim}
\end{tcolorbox}

            \begin{tcolorbox}[breakable, size=fbox, boxrule=.5pt, pad at break*=1mm, opacityfill=0]
\prompt{Out}{outcolor}{8}{\boxspacing}
\begin{Verbatim}[commandchars=\\\{\}]
[1960, 1961, 1962, 1963, 1964, 1965, 1966, 1967, 1968]
\end{Verbatim}
\end{tcolorbox}
        
    \textbf{Uwaga!} Tu łatwo przeoczyć, że całość \texttt{for} jest w
nawiasach kwadratowych, co dla osoby początkującej może wydawać się
dziwne - zamiast wartości tablicy mamy kod. Przyjrzyj się dokładniej.
Zwróć uwagę, że są to nawiasy kwadratowe i takie coś:
\texttt{print(year\ for\ year\ in\ years)} nie działa.

    \begin{tcolorbox}[breakable, size=fbox, boxrule=1pt, pad at break*=1mm,colback=cellbackground, colframe=cellborder]
\prompt{In}{incolor}{9}{\boxspacing}
\begin{Verbatim}[commandchars=\\\{\}]
\PY{n}{output} \PY{o}{=} \PY{p}{[}\PY{p}{]}
\PY{k}{for} \PY{n}{year} \PY{o+ow}{in} \PY{n}{years}\PY{p}{:}
    \PY{k}{if} \PY{n}{year} \PY{o}{\PYZpc{}} \PY{l+m+mi}{2} \PY{o}{==} \PY{l+m+mi}{0}\PY{p}{:} 
        \PY{n}{output}\PY{o}{.}\PY{n}{append}\PY{p}{(}\PY{n}{year}\PY{o}{*}\PY{l+m+mi}{2}\PY{p}{)}
\PY{n}{output}  
\end{Verbatim}
\end{tcolorbox}

            \begin{tcolorbox}[breakable, size=fbox, boxrule=.5pt, pad at break*=1mm, opacityfill=0]
\prompt{Out}{outcolor}{9}{\boxspacing}
\begin{Verbatim}[commandchars=\\\{\}]
[4020, 4024, 4028, 4032, 4036]
\end{Verbatim}
\end{tcolorbox}
        
    \hypertarget{numpy}{%
\subsection{\texorpdfstring{\href{https://bit.ly/3u2hM8m}{numpy}}{numpy}}\label{numpy}}

To biblioteka do pracy z wektorami lub macierzami. Jest bardzo szybka (w
porównaniu do ``czystego'' Pythona) i dlatego będziemy jej używać.

    \begin{tcolorbox}[breakable, size=fbox, boxrule=1pt, pad at break*=1mm,colback=cellbackground, colframe=cellborder]
\prompt{In}{incolor}{10}{\boxspacing}
\begin{Verbatim}[commandchars=\\\{\}]
\PY{k+kn}{import} \PY{n+nn}{numpy} \PY{k}{as} \PY{n+nn}{np} \PY{c+c1}{\PYZsh{}przyjęło się, że dla numpy jest nadawy alias \PYZdq{}np\PYZdq{}}
\end{Verbatim}
\end{tcolorbox}

    \begin{tcolorbox}[breakable, size=fbox, boxrule=1pt, pad at break*=1mm,colback=cellbackground, colframe=cellborder]
\prompt{In}{incolor}{11}{\boxspacing}
\begin{Verbatim}[commandchars=\\\{\}]
\PY{n}{arr} \PY{o}{=} \PY{n}{np}\PY{o}{.}\PY{n}{array}\PY{p}{(}\PY{p}{[}\PY{l+m+mi}{1}\PY{p}{,} \PY{l+m+mi}{2}\PY{p}{,} \PY{l+m+mi}{3}\PY{p}{,} \PY{l+m+mi}{4}\PY{p}{,} \PY{l+m+mi}{5}\PY{p}{,} \PY{l+m+mi}{6}\PY{p}{]}\PY{p}{)} \PY{c+c1}{\PYZsh{} tablica}
\PY{n}{arr}
\end{Verbatim}
\end{tcolorbox}

            \begin{tcolorbox}[breakable, size=fbox, boxrule=.5pt, pad at break*=1mm, opacityfill=0]
\prompt{Out}{outcolor}{11}{\boxspacing}
\begin{Verbatim}[commandchars=\\\{\}]
array([1, 2, 3, 4, 5, 6])
\end{Verbatim}
\end{tcolorbox}
        
    \begin{tcolorbox}[breakable, size=fbox, boxrule=1pt, pad at break*=1mm,colback=cellbackground, colframe=cellborder]
\prompt{In}{incolor}{12}{\boxspacing}
\begin{Verbatim}[commandchars=\\\{\}]
\PY{n+nb}{type}\PY{p}{(}\PY{n}{arr}\PY{p}{)}
\end{Verbatim}
\end{tcolorbox}

            \begin{tcolorbox}[breakable, size=fbox, boxrule=.5pt, pad at break*=1mm, opacityfill=0]
\prompt{Out}{outcolor}{12}{\boxspacing}
\begin{Verbatim}[commandchars=\\\{\}]
numpy.ndarray
\end{Verbatim}
\end{tcolorbox}
        
    \begin{tcolorbox}[breakable, size=fbox, boxrule=1pt, pad at break*=1mm,colback=cellbackground, colframe=cellborder]
\prompt{In}{incolor}{13}{\boxspacing}
\begin{Verbatim}[commandchars=\\\{\}]
\PY{n}{np}\PY{o}{.}\PY{n}{array}\PY{p}{(}\PY{p}{[}\PY{l+m+mi}{1}\PY{p}{,} \PY{l+m+mi}{2}\PY{p}{,} \PY{l+m+mi}{3}\PY{p}{]}\PY{p}{)} \PY{o}{+} \PY{n}{np}\PY{o}{.}\PY{n}{array}\PY{p}{(}\PY{p}{[}\PY{l+m+mi}{5}\PY{p}{,} \PY{l+m+mi}{6}\PY{p}{,} \PY{l+m+mi}{7}\PY{p}{]}\PY{p}{)}
\end{Verbatim}
\end{tcolorbox}

            \begin{tcolorbox}[breakable, size=fbox, boxrule=.5pt, pad at break*=1mm, opacityfill=0]
\prompt{Out}{outcolor}{13}{\boxspacing}
\begin{Verbatim}[commandchars=\\\{\}]
array([ 6,  8, 10])
\end{Verbatim}
\end{tcolorbox}
        
    \begin{tcolorbox}[breakable, size=fbox, boxrule=1pt, pad at break*=1mm,colback=cellbackground, colframe=cellborder]
\prompt{In}{incolor}{14}{\boxspacing}
\begin{Verbatim}[commandchars=\\\{\}]
\PY{n}{np}\PY{o}{.}\PY{n}{mean}\PY{p}{(}\PY{p}{[}\PY{l+m+mi}{1}\PY{p}{,} \PY{l+m+mi}{2}\PY{p}{,} \PY{l+m+mi}{3}\PY{p}{]}\PY{p}{)}
\end{Verbatim}
\end{tcolorbox}

            \begin{tcolorbox}[breakable, size=fbox, boxrule=.5pt, pad at break*=1mm, opacityfill=0]
\prompt{Out}{outcolor}{14}{\boxspacing}
\begin{Verbatim}[commandchars=\\\{\}]
2.0
\end{Verbatim}
\end{tcolorbox}
        
    \begin{tcolorbox}[breakable, size=fbox, boxrule=1pt, pad at break*=1mm,colback=cellbackground, colframe=cellborder]
\prompt{In}{incolor}{15}{\boxspacing}
\begin{Verbatim}[commandchars=\\\{\}]
\PY{n}{np}\PY{o}{.}\PY{n}{array}\PY{p}{(}\PY{p}{[}\PY{l+m+mi}{1}\PY{p}{,} \PY{l+m+mi}{2}\PY{p}{,} \PY{l+m+mi}{3}\PY{p}{]}\PY{p}{)}\PY{o}{.}\PY{n}{mean}\PY{p}{(}\PY{p}{)}
\end{Verbatim}
\end{tcolorbox}

            \begin{tcolorbox}[breakable, size=fbox, boxrule=.5pt, pad at break*=1mm, opacityfill=0]
\prompt{Out}{outcolor}{15}{\boxspacing}
\begin{Verbatim}[commandchars=\\\{\}]
2.0
\end{Verbatim}
\end{tcolorbox}
        
    \begin{tcolorbox}[breakable, size=fbox, boxrule=1pt, pad at break*=1mm,colback=cellbackground, colframe=cellborder]
\prompt{In}{incolor}{16}{\boxspacing}
\begin{Verbatim}[commandchars=\\\{\}]
\PY{n+nb}{print}\PY{p}{(}\PY{n}{arr} \PY{o}{==} \PY{l+m+mi}{1}\PY{p}{)} \PY{c+c1}{\PYZsh{}tylko pierwszy element jest True (bo jest równy 1)}
\PY{n+nb}{print}\PY{p}{(}\PY{n}{arr} \PY{o}{==} \PY{l+m+mi}{2}\PY{p}{)} \PY{c+c1}{\PYZsh{}tylko drugie element jest True (bo jest równy 2)}
\PY{n+nb}{print}\PY{p}{(}\PY{n}{arr} \PY{o}{\PYZgt{}} \PY{l+m+mi}{2}\PY{p}{)} \PY{c+c1}{\PYZsh{}pierwsze dwa nie pasują (tylko 3 i wyżej są większe od 2), reszt pasuję}
\PY{n+nb}{print}\PY{p}{(}\PY{n}{arr} \PY{o}{\PYZpc{}}\PY{k}{2} == 0) \PYZsh{} tylko parzyste
\end{Verbatim}
\end{tcolorbox}

    \begin{Verbatim}[commandchars=\\\{\}]
[ True False False False False False]
[False  True False False False False]
[False False  True  True  True  True]
[False  True False  True False  True]
    \end{Verbatim}

    Zostawiamy tylko parzyste elementy w tablicy. Operator ``\%'' to jest
\href{https://bit.ly/39njYPS}{modulo}.

    \begin{tcolorbox}[breakable, size=fbox, boxrule=1pt, pad at break*=1mm,colback=cellbackground, colframe=cellborder]
\prompt{In}{incolor}{17}{\boxspacing}
\begin{Verbatim}[commandchars=\\\{\}]
\PY{n}{arr}\PY{p}{[} \PY{n}{arr} \PY{o}{\PYZpc{}} \PY{l+m+mi}{2} \PY{o}{==} \PY{l+m+mi}{0} \PY{p}{]}
\end{Verbatim}
\end{tcolorbox}

            \begin{tcolorbox}[breakable, size=fbox, boxrule=.5pt, pad at break*=1mm, opacityfill=0]
\prompt{Out}{outcolor}{17}{\boxspacing}
\begin{Verbatim}[commandchars=\\\{\}]
array([2, 4, 6])
\end{Verbatim}
\end{tcolorbox}
        
    Na samym dole strony są dwa linki do 2-godzinnych webinarów na temat:
python/numpy + pandas. Są tam również przykłady na githubie. Zapoznaj
się z tym i spróbuj to wykonać samodzielnie (przynajmniej część).

    \hypertarget{zadania-domowe}{%
\section{Zadania domowe}\label{zadania-domowe}}

    \hypertarget{zadanie-1.2.1}{%
\subsection{Zadanie 1.2.1}\label{zadanie-1.2.1}}

Masz listę od 0 do 18. Twoim zadaniem jest zostawić tylko liczby, które
dzielą się na 4 bez reszty (czyli 0, 4, 8, 12 i 16).

    \begin{tcolorbox}[breakable, size=fbox, boxrule=1pt, pad at break*=1mm,colback=cellbackground, colframe=cellborder]
\prompt{In}{incolor}{21}{\boxspacing}
\begin{Verbatim}[commandchars=\\\{\}]
\PY{n}{numbers} \PY{o}{=} \PY{n+nb}{range}\PY{p}{(}\PY{l+m+mi}{19}\PY{p}{)}
\PY{p}{[} \PY{n}{x} \PY{k}{for} \PY{n}{x} \PY{o+ow}{in} \PY{n}{numbers} \PY{k}{if} \PY{n}{x} \PY{o}{\PYZpc{}} \PY{l+m+mi}{4} \PY{o}{==} \PY{l+m+mi}{0}\PY{p}{]}
\end{Verbatim}
\end{tcolorbox}

            \begin{tcolorbox}[breakable, size=fbox, boxrule=.5pt, pad at break*=1mm, opacityfill=0]
\prompt{Out}{outcolor}{21}{\boxspacing}
\begin{Verbatim}[commandchars=\\\{\}]
[0, 4, 8, 12, 16]
\end{Verbatim}
\end{tcolorbox}
        
    👉 Kliknij tutaj (1 klik), aby zobaczyć podpowiedź 👈

Pamiętaj, że możesz to napisać w czystym Python (spróbuj), ale do
bardziej eleganckiego rozwiązania warto użyć \texttt{numpy}.

👉 Kliknij tutaj (1 klik), aby zobaczyć odpowiedź 👈

\begin{Shaded}
\begin{Highlighting}[]
\NormalTok{my\_arr }\OperatorTok{=} \BuiltInTok{range}\NormalTok{(}\DecValTok{19}\NormalTok{)}
\NormalTok{[ x }\ControlFlowTok{for}\NormalTok{ x }\KeywordTok{in}\NormalTok{ my\_arr }\ControlFlowTok{if}\NormalTok{ x }\OperatorTok{\%} \DecValTok{4} \OperatorTok{==} \DecValTok{0}\NormalTok{]}
    
\CommentTok{\#numpy}
\NormalTok{my\_arr }\OperatorTok{=}\NormalTok{ np.array(my\_arr)}
\NormalTok{my\_arr[ my\_arr }\OperatorTok{\%}\DecValTok{4} \OperatorTok{==} \DecValTok{0}\NormalTok{ ]}
\end{Highlighting}
\end{Shaded}

Pamiętaj, aby zerkać tutaj dopiero, gdy już spróbujesz wykonać zadanie
samodzielnie.

    \hypertarget{wspuxf3ux142praca-i-komunikacja}{%
\subsubsection{🤝🗣️ Współpraca 💪 i komunikacja
💬}\label{wspuxf3ux142praca-i-komunikacja}}

\begin{itemize}
\tightlist
\item
  👉
  \href{https://practicalmlcourse.slack.com/archives/C045CNLNH89}{\#pml\_module1}
  - to jest miejsce, gdzie można szukać pomocy i dzielić się
  doświadczeniem - także pomagać innym 🥰.
\end{itemize}

Jeśli masz pytanie, to staraj się jak najdokładniej je sprecyzować,
najlepiej wrzuć screen z twoim kodem i błędem, który się pojawił ✔️

\begin{itemize}
\item
  👉
  \href{https://practicalmlcourse.slack.com/archives/C045CP89KND}{\#pml\_module1\_done}
  - to miejsce, gdzie możesz dzielić się swoimi przerobionymi zadaniami,
  wystarczy, że wrzucisz screen z \#done i numerem lekcji np.
  \emph{\#1.2.1\_done}, śmiało dodaj komentarz, jeśli czujesz taką
  potrzebę, a także rozmawiaj z innymi o ich rozwiązaniach 😊
\item
  👉
  \href{https://practicalmlcourse.slack.com/archives/C044TFZLF1U}{\#pml\_module1\_ideas}-
  tutaj możesz dzielić się swoimi pomysłami
\end{itemize}

    \hypertarget{zadanie-1.2.2}{%
\subsection{Zadanie 1.2.2}\label{zadanie-1.2.2}}

Masz dwie listy i chcesz zsumować każdą parę (np. pierwszy element
list\_c{[}0{]} = list\_a{[}0{]} + list\_b{[}0{]}), w wyniku dostaniesz
trzecią listę (\texttt{list\_c}). Następnie chcesz znaleźć medianę
trzeciej listy (czyli \texttt{median\_list\_c}).

    \begin{tcolorbox}[breakable, size=fbox, boxrule=1pt, pad at break*=1mm,colback=cellbackground, colframe=cellborder]
\prompt{In}{incolor}{18}{\boxspacing}
\begin{Verbatim}[commandchars=\\\{\}]
\PY{k+kn}{import} \PY{n+nn}{numpy} \PY{k}{as} \PY{n+nn}{np}

\PY{n}{list\PYZus{}a} \PY{o}{=} \PY{n}{np}\PY{o}{.}\PY{n}{array}\PY{p}{(}\PY{n+nb}{range}\PY{p}{(}\PY{l+m+mi}{10}\PY{p}{)}\PY{p}{)}
\PY{n}{list\PYZus{}b} \PY{o}{=} \PY{n}{np}\PY{o}{.}\PY{n}{array}\PY{p}{(}\PY{n+nb}{range}\PY{p}{(}\PY{l+m+mi}{10}\PY{p}{,}\PY{l+m+mi}{20}\PY{p}{)}\PY{p}{)}
\PY{n}{list\PYZus{}c} \PY{o}{=} \PY{n}{list\PYZus{}a} \PY{o}{+} \PY{n}{list\PYZus{}b}
\PY{n+nb}{print}\PY{p}{(}\PY{n}{list\PYZus{}c}\PY{p}{)}
\PY{n}{median\PYZus{}list\PYZus{}c} \PY{o}{=} \PY{n}{np}\PY{o}{.}\PY{n}{median}\PY{p}{(}\PY{n}{list\PYZus{}c}\PY{p}{)}
\PY{n+nb}{print}\PY{p}{(}\PY{n}{median\PYZus{}list\PYZus{}c}\PY{p}{)}
\end{Verbatim}
\end{tcolorbox}

    \begin{Verbatim}[commandchars=\\\{\}]
[10 12 14 16 18 20 22 24 26 28]
19.0
    \end{Verbatim}

    👉 Kliknij tutaj (1 klik), aby zobaczyć podpowiedź 👈

Utwórz dwie listy (o tej samej długości) i użyj funkcji np.median

👉 Kliknij tutaj (1 klik), aby zobaczyć odpowiedź 👈

\begin{Shaded}
\begin{Highlighting}[]
\NormalTok{list\_a }\OperatorTok{=}\NormalTok{ np.arange(}\DecValTok{10}\NormalTok{)}
\NormalTok{list\_b }\OperatorTok{=}\NormalTok{ np.arange(}\DecValTok{10}\NormalTok{) }\OperatorTok{*} \DecValTok{2}

\NormalTok{np.median(list\_a }\OperatorTok{+}\NormalTok{ list\_b)}
\end{Highlighting}
\end{Shaded}

Najpierw próbujesz zrobić zadanie samodzielnie, a dopiero potem tutaj
zerkasz, prawda? ;)

    \hypertarget{zadanie-1.2.3}{%
\subsection{Zadanie 1.2.3}\label{zadanie-1.2.3}}

Masz listę, która posiada 1000 elementów. Chcesz zostawić wszystkie
elementy, które mają indeksy od 50 do 99 i od 325 do 388 (włącznie).

    \begin{tcolorbox}[breakable, size=fbox, boxrule=1pt, pad at break*=1mm,colback=cellbackground, colframe=cellborder]
\prompt{In}{incolor}{19}{\boxspacing}
\begin{Verbatim}[commandchars=\\\{\}]
\PY{k+kn}{import} \PY{n+nn}{random}

\PY{n}{randomlist} \PY{o}{=} \PY{p}{[}\PY{p}{]}
\PY{k}{for} \PY{n}{i} \PY{o+ow}{in} \PY{n+nb}{range}\PY{p}{(}\PY{l+m+mi}{1000}\PY{p}{)}\PY{p}{:}
    \PY{n}{n} \PY{o}{=} \PY{n}{random}\PY{o}{.}\PY{n}{randint}\PY{p}{(}\PY{l+m+mi}{1}\PY{p}{,}\PY{l+m+mi}{100}\PY{p}{)}
    \PY{n}{randomlist}\PY{o}{.}\PY{n}{append}\PY{p}{(}\PY{n}{n}\PY{p}{)}
    
\PY{n+nb}{print}\PY{p}{(}\PY{n}{randomlist}\PY{p}{)}

\PY{n}{list1} \PY{o}{=} \PY{n}{randomlist}\PY{p}{[}\PY{l+m+mi}{50}\PY{p}{:}\PY{l+m+mi}{99}\PY{p}{]}
\PY{n}{list2} \PY{o}{=} \PY{n}{randomlist}\PY{p}{[}\PY{l+m+mi}{325}\PY{p}{:}\PY{l+m+mi}{388}\PY{p}{]}

\PY{n+nb}{print}\PY{p}{(}\PY{l+s+s2}{\PYZdq{}}\PY{l+s+s2}{Output:}\PY{l+s+s2}{\PYZdq{}}\PY{p}{,} \PY{n}{list1} \PY{o}{+} \PY{n}{list2}\PY{p}{,} \PY{n}{sep}\PY{o}{=}\PY{l+s+s1}{\PYZsq{}}\PY{l+s+se}{\PYZbs{}n}\PY{l+s+s1}{\PYZsq{}}\PY{p}{)}
\end{Verbatim}
\end{tcolorbox}

    \begin{Verbatim}[commandchars=\\\{\}]
[16, 71, 89, 77, 84, 3, 63, 76, 34, 98, 28, 25, 42, 55, 20, 62, 75, 46, 95, 85,
73, 58, 93, 89, 16, 20, 74, 98, 43, 11, 8, 97, 54, 24, 68, 94, 54, 10, 77, 32,
38, 3, 77, 89, 88, 100, 26, 45, 6, 21, 45, 70, 32, 33, 72, 5, 29, 86, 36, 88,
31, 40, 72, 90, 66, 63, 24, 40, 21, 71, 41, 88, 16, 80, 34, 54, 80, 97, 54, 95,
88, 83, 45, 88, 10, 24, 63, 2, 8, 35, 100, 87, 55, 47, 86, 11, 18, 82, 82, 91,
32, 81, 56, 23, 74, 74, 54, 71, 84, 86, 14, 10, 66, 3, 52, 63, 50, 26, 79, 7,
86, 58, 36, 10, 93, 84, 54, 83, 91, 83, 5, 90, 29, 76, 7, 15, 9, 54, 5, 27, 22,
79, 11, 9, 84, 51, 14, 42, 63, 51, 76, 72, 83, 1, 46, 61, 69, 67, 68, 85, 91,
20, 100, 99, 100, 28, 100, 19, 88, 39, 98, 42, 18, 26, 66, 93, 4, 65, 94, 20,
90, 28, 40, 77, 85, 62, 44, 75, 93, 65, 32, 98, 10, 20, 88, 21, 19, 84, 3, 83,
66, 66, 81, 36, 97, 60, 27, 98, 50, 34, 47, 85, 54, 60, 12, 56, 43, 90, 10, 71,
39, 67, 77, 33, 92, 14, 84, 24, 62, 95, 53, 50, 9, 26, 46, 20, 25, 85, 25, 15,
73, 25, 16, 33, 60, 31, 76, 28, 93, 71, 42, 97, 16, 77, 77, 7, 44, 78, 36, 23,
69, 40, 25, 7, 8, 92, 81, 24, 18, 5, 13, 72, 37, 55, 14, 96, 85, 6, 42, 79, 88,
27, 59, 67, 78, 63, 44, 19, 83, 10, 45, 47, 85, 31, 89, 47, 49, 23, 76, 42, 28,
92, 96, 80, 66, 86, 59, 90, 73, 6, 62, 36, 57, 6, 5, 32, 20, 82, 37, 21, 94, 74,
37, 11, 7, 36, 49, 1, 97, 27, 81, 2, 3, 2, 18, 42, 23, 21, 41, 59, 79, 86, 14,
35, 17, 68, 9, 8, 40, 80, 78, 26, 26, 89, 13, 86, 83, 98, 72, 44, 44, 17, 66,
23, 22, 12, 6, 86, 79, 32, 26, 80, 71, 93, 31, 4, 48, 47, 71, 84, 19, 52, 19,
52, 72, 16, 51, 1, 51, 50, 45, 99, 52, 8, 64, 44, 59, 3, 17, 72, 24, 86, 81, 23,
61, 4, 8, 44, 65, 41, 67, 14, 2, 41, 81, 27, 66, 28, 67, 10, 35, 39, 71, 22, 14,
74, 40, 71, 14, 2, 83, 71, 17, 16, 14, 87, 40, 87, 3, 97, 88, 80, 40, 20, 62,
37, 51, 54, 6, 35, 7, 76, 46, 68, 100, 28, 35, 84, 97, 86, 65, 44, 45, 66, 76,
97, 19, 29, 18, 87, 90, 70, 49, 78, 79, 26, 7, 39, 10, 11, 32, 85, 38, 19, 7, 7,
68, 33, 93, 10, 4, 81, 25, 85, 47, 28, 54, 46, 28, 63, 81, 80, 67, 75, 92, 2,
32, 57, 8, 15, 6, 72, 97, 45, 40, 35, 55, 73, 54, 64, 64, 46, 58, 21, 47, 33,
74, 5, 96, 99, 94, 80, 37, 54, 38, 86, 67, 60, 1, 40, 19, 13, 30, 88, 32, 1, 63,
48, 18, 71, 87, 19, 1, 7, 86, 37, 47, 51, 57, 64, 84, 58, 29, 82, 84, 35, 74,
45, 11, 51, 76, 41, 61, 15, 75, 51, 23, 92, 98, 11, 40, 6, 94, 95, 53, 22, 10,
19, 26, 61, 10, 63, 76, 52, 96, 78, 76, 65, 49, 10, 10, 47, 73, 87, 10, 52, 64,
32, 66, 52, 7, 69, 73, 30, 91, 92, 61, 59, 27, 93, 7, 100, 29, 76, 32, 61, 34,
74, 72, 21, 36, 56, 99, 42, 48, 93, 4, 50, 63, 37, 97, 18, 83, 99, 90, 74, 94,
1, 55, 33, 16, 46, 87, 78, 65, 33, 59, 58, 70, 33, 67, 16, 44, 56, 54, 39, 20,
24, 2, 77, 73, 75, 71, 70, 10, 12, 90, 28, 67, 100, 19, 26, 82, 29, 7, 59, 47,
12, 53, 12, 61, 4, 86, 36, 96, 52, 75, 20, 71, 33, 13, 28, 89, 67, 84, 75, 21,
62, 57, 6, 77, 66, 70, 4, 73, 87, 1, 14, 62, 30, 86, 12, 53, 54, 67, 8, 10, 96,
71, 72, 60, 42, 8, 72, 57, 12, 81, 44, 73, 18, 55, 90, 88, 69, 5, 58, 72, 72,
90, 40, 98, 56, 31, 45, 25, 12, 12, 27, 32, 53, 24, 59, 91, 3, 56, 7, 67, 46, 4,
38, 29, 18, 62, 25, 77, 62, 87, 92, 4, 11, 49, 15, 98, 80, 75, 21, 94, 56, 2,
78, 63, 4, 20, 66, 92, 48, 94, 91, 27, 98, 4, 7, 17, 96, 53, 34, 54, 82, 42, 5,
47, 63, 45, 96, 17, 45, 66, 48, 17, 84, 89, 31, 31, 70, 88, 41, 70, 69, 55, 40,
80, 30, 59, 17, 76, 78, 84, 21, 48, 1, 42, 85, 8, 91, 72, 46, 17, 44, 14, 72,
20, 96, 67, 66, 91, 31, 11, 91, 15, 79, 92, 45, 44, 7, 54, 48, 49, 89, 83, 35,
1, 48, 13, 4, 30, 88, 50, 1, 61, 61, 1, 20, 78, 86, 98, 25, 51, 62, 4, 21, 10,
45, 41, 20, 29, 15, 89, 55, 81, 18, 27, 71, 29, 78, 8, 26, 63, 88, 66, 81, 36,
14, 48, 70, 70, 79, 90, 32, 89, 53, 19, 46, 24, 95, 68, 93, 51, 44, 16, 86, 68,
73, 13, 99, 44, 98, 17, 33, 96, 16, 32, 48, 85, 16, 45, 99, 59, 85, 70, 66, 89,
49, 78, 27, 12, 4, 4, 29, 37, 87, 16, 88, 87, 14, 67, 76, 40, 26, 65, 93, 65,
98, 16, 78, 62, 17, 88, 90, 66, 23, 16, 72, 73, 65, 98, 10, 53, 53, 68, 23, 9,
85, 57, 77, 80, 49, 84, 91, 12, 62]
Output:
[45, 70, 32, 33, 72, 5, 29, 86, 36, 88, 31, 40, 72, 90, 66, 63, 24, 40, 21, 71,
41, 88, 16, 80, 34, 54, 80, 97, 54, 95, 88, 83, 45, 88, 10, 24, 63, 2, 8, 35,
100, 87, 55, 47, 86, 11, 18, 82, 82, 36, 49, 1, 97, 27, 81, 2, 3, 2, 18, 42, 23,
21, 41, 59, 79, 86, 14, 35, 17, 68, 9, 8, 40, 80, 78, 26, 26, 89, 13, 86, 83,
98, 72, 44, 44, 17, 66, 23, 22, 12, 6, 86, 79, 32, 26, 80, 71, 93, 31, 4, 48,
47, 71, 84, 19, 52, 19, 52, 72, 16, 51, 1]
    \end{Verbatim}

    \#

👉 Kliknij tutaj (1 klik), aby zobaczyć podpowiedź 👈

Należy użyć łącznie 4 filtrów (po dwie grupy, wewnątrz operatora i i na
zewnątrz, między grupami, operator)

👉 Kliknij tutaj (1 klik), aby zobaczyć odpowiedź 👈

\begin{Shaded}
\begin{Highlighting}[]
\NormalTok{my\_list }\OperatorTok{=}\NormalTok{ np.arange(}\DecValTok{1000}\NormalTok{)}


\NormalTok{frst\_ }\OperatorTok{=}\NormalTok{ (my\_list }\OperatorTok{\textgreater{}=} \DecValTok{50}\NormalTok{) }\OperatorTok{\&}\NormalTok{ (my\_list }\OperatorTok{\textless{}} \DecValTok{100}\NormalTok{)}
\NormalTok{scnd\_ }\OperatorTok{=}\NormalTok{ (my\_list }\OperatorTok{\textgreater{}=} \DecValTok{325}\NormalTok{) }\OperatorTok{\&}\NormalTok{ (my\_list }\OperatorTok{\textless{}} \DecValTok{389}\NormalTok{)}
\NormalTok{my\_list[  frst\_ }\OperatorTok{|}\NormalTok{ scnd\_ ]}
\end{Highlighting}
\end{Shaded}

    \hypertarget{pandas}{%
\subsection{\texorpdfstring{\href{https://bit.ly/3fmipW5}{pandas}}{pandas}}\label{pandas}}

To biblioteka, która będzie nam bardzo potrzebna dalej. O
\texttt{pandas} możesz myśleć jak o \texttt{numpy} na ``sterydach'' albo
jak o tabelce z danymi jak w Excelu, w którym możesz w bardzo łatwy
sposób filtrować i grupować dane, wybierać potrzebne kolumny itd.

    \begin{tcolorbox}[breakable, size=fbox, boxrule=1pt, pad at break*=1mm,colback=cellbackground, colframe=cellborder]
\prompt{In}{incolor}{21}{\boxspacing}
\begin{Verbatim}[commandchars=\\\{\}]
\PY{k+kn}{import} \PY{n+nn}{pandas} \PY{k}{as} \PY{n+nn}{pd} \PY{c+c1}{\PYZsh{}przyjęło się nadawać dla pandas alias \PYZdq{}pd\PYZdq{}}
\end{Verbatim}
\end{tcolorbox}

    \texttt{Pandas} wewnątrz używa \texttt{numpy}, dlatego składnia na
przykład filtrowania, jest podobna (lub czasem identyczna). Bardzo łatwo
można także konwertować dane z \texttt{pandas} do \texttt{numpy}.

W terminologii \texttt{numpy} wszystko obraca się wokół wektorów i
macierzy. W pandas mamy podobnie dwie struktury: \texttt{Series}
(odpowiednik wektora) i \texttt{DataFrame} (odpowiednik macierzy).

Oczywiście \texttt{DataFrame} jest bardziej rozbudowany niż goła
macierz. Zobacz poniżej.

    \begin{tcolorbox}[breakable, size=fbox, boxrule=1pt, pad at break*=1mm,colback=cellbackground, colframe=cellborder]
\prompt{In}{incolor}{22}{\boxspacing}
\begin{Verbatim}[commandchars=\\\{\}]
\PY{n}{np}\PY{o}{.}\PY{n}{random}\PY{o}{.}\PY{n}{seed}\PY{p}{(}\PY{l+m+mi}{2018}\PY{p}{)}
\PY{n}{rows} \PY{o}{=} \PY{l+m+mi}{15}

\PY{n}{df} \PY{o}{=} \PY{n}{pd}\PY{o}{.}\PY{n}{DataFrame}\PY{p}{(}\PY{p}{\PYZob{}}
    \PY{l+s+s1}{\PYZsq{}}\PY{l+s+s1}{height}\PY{l+s+s1}{\PYZsq{}}\PY{p}{:} \PY{n}{np}\PY{o}{.}\PY{n}{random}\PY{o}{.}\PY{n}{randint}\PY{p}{(}\PY{l+m+mi}{140}\PY{p}{,} \PY{l+m+mi}{210}\PY{p}{,} \PY{n}{rows}\PY{p}{)}\PY{p}{,}
    \PY{l+s+s1}{\PYZsq{}}\PY{l+s+s1}{weight}\PY{l+s+s1}{\PYZsq{}}\PY{p}{:} \PY{n}{np}\PY{o}{.}\PY{n}{random}\PY{o}{.}\PY{n}{randint}\PY{p}{(}\PY{l+m+mi}{50}\PY{p}{,} \PY{l+m+mi}{80}\PY{p}{,} \PY{n}{rows}\PY{p}{)}\PY{p}{,}
\PY{p}{\PYZcb{}}\PY{p}{)}

\PY{n}{df}
\end{Verbatim}
\end{tcolorbox}

            \begin{tcolorbox}[breakable, size=fbox, boxrule=.5pt, pad at break*=1mm, opacityfill=0]
\prompt{Out}{outcolor}{22}{\boxspacing}
\begin{Verbatim}[commandchars=\\\{\}]
    height  weight
0      149      76
1      161      56
2      168      63
3      160      71
4      146      72
5      165      73
6      165      50
7      187      73
8      165      58
9      200      57
10     171      59
11     162      70
12     146      58
13     182      67
14     140      77
\end{Verbatim}
\end{tcolorbox}
        
    Dostać się do danych w wybranej kolumnie można na co najmniej dwa
sposoby.

    \begin{tcolorbox}[breakable, size=fbox, boxrule=1pt, pad at break*=1mm,colback=cellbackground, colframe=cellborder]
\prompt{In}{incolor}{23}{\boxspacing}
\begin{Verbatim}[commandchars=\\\{\}]
\PY{n}{df}\PY{o}{.}\PY{n}{height}

\PY{c+c1}{\PYZsh{}lub}

\PY{n}{df}\PY{p}{[}\PY{l+s+s1}{\PYZsq{}}\PY{l+s+s1}{height}\PY{l+s+s1}{\PYZsq{}}\PY{p}{]}
\end{Verbatim}
\end{tcolorbox}

            \begin{tcolorbox}[breakable, size=fbox, boxrule=.5pt, pad at break*=1mm, opacityfill=0]
\prompt{Out}{outcolor}{23}{\boxspacing}
\begin{Verbatim}[commandchars=\\\{\}]
0     149
1     161
2     168
3     160
4     146
5     165
6     165
7     187
8     165
9     200
10    171
11    162
12    146
13    182
14    140
Name: height, dtype: int64
\end{Verbatim}
\end{tcolorbox}
        
    Dość często używany jest pierwszy sposób, bo ma mniej znaków (nie trzeba
otwierać zamykać nawiasów kwadratowych i apostrofów). Natomiast druga
wersja ma taką przewagę, że jest bardziej stabilna (np. jeśli kolumna
nazywa się ``count'' to pierwsza wersja pomyli nazwę kolumny z nazwą
funkcji \texttt{count()}, to samo dotyczy \texttt{min}, \texttt{max}
itd). Sprawdźmy to.

    \begin{tcolorbox}[breakable, size=fbox, boxrule=1pt, pad at break*=1mm,colback=cellbackground, colframe=cellborder]
\prompt{In}{incolor}{24}{\boxspacing}
\begin{Verbatim}[commandchars=\\\{\}]
\PY{n}{df}\PY{p}{[}\PY{l+s+s1}{\PYZsq{}}\PY{l+s+s1}{count}\PY{l+s+s1}{\PYZsq{}}\PY{p}{]} \PY{o}{=} \PY{l+m+mi}{1}

\PY{n+nb}{print}\PY{p}{(}\PY{n+nb}{type}\PY{p}{(}\PY{n}{df}\PY{o}{.}\PY{n}{count}\PY{p}{)}\PY{p}{)}
\PY{n+nb}{print}\PY{p}{(}\PY{n+nb}{type}\PY{p}{(}\PY{n}{df}\PY{o}{.}\PY{n}{height}\PY{p}{)}\PY{p}{)}
\end{Verbatim}
\end{tcolorbox}

    \begin{Verbatim}[commandchars=\\\{\}]
<class 'method'>
<class 'pandas.core.series.Series'>
    \end{Verbatim}

    Przeanalizujmy to krok po kroku.

Po pierwsze: stworzyliśmy nową kolumnę w \texttt{dataframe}, w naszym
przypadku \texttt{df} o nazwie \texttt{count}. Jak widzisz, stworzenie
nowej kolumny wymaga użycia drugiego sposobu (czyli dłuższego). Jeśli
spróbujesz stworzyć kolumnę w ten sposób: \texttt{df.count\ =\ 1}, to
kolumna się nie utworzy. Sprawdź :).

Po drugie: możemy przypisać do kolumny stałą. W naszym przypadku to była
1. Czyli wszystkie wiersze w kolumnie \texttt{count} mają wartość 1.

Po trzecie: spróbujmy odwołać się do kolumn poprzez kropkę. Widzimy, że
w przypadku z \texttt{height} to zadziałało, bo nie ma konfliktów w
nazwie. Natomiast dla \texttt{count} to nie przeszło, bo uznało, że
chodzi o funkcję o nazwie \texttt{count}, która zlicza ilość zamiast
odwołać się do kolumny o nazwie \texttt{count}, dlatego należy
podchodzić do tego ostrożnie.

Stwórzmy jeszcze jedną kolumnę bazując na dwóch obecnych. Możemy np.
podzielić wzrost przez wagę.

    \begin{tcolorbox}[breakable, size=fbox, boxrule=1pt, pad at break*=1mm,colback=cellbackground, colframe=cellborder]
\prompt{In}{incolor}{25}{\boxspacing}
\begin{Verbatim}[commandchars=\\\{\}]
\PY{n}{df}\PY{p}{[}\PY{l+s+s1}{\PYZsq{}}\PY{l+s+s1}{ratio}\PY{l+s+s1}{\PYZsq{}}\PY{p}{]} \PY{o}{=} \PY{n}{df}\PY{p}{[}\PY{l+s+s1}{\PYZsq{}}\PY{l+s+s1}{height}\PY{l+s+s1}{\PYZsq{}}\PY{p}{]} \PY{o}{/} \PY{n}{df}\PY{p}{[}\PY{l+s+s1}{\PYZsq{}}\PY{l+s+s1}{weight}\PY{l+s+s1}{\PYZsq{}}\PY{p}{]}

\PY{c+c1}{\PYZsh{}lub}
\PY{c+c1}{\PYZsh{}df[\PYZsq{}ratio\PYZsq{}] = df.height / df.weight}

\PY{c+c1}{\PYZsh{}źle (nie zadziała)}
\PY{c+c1}{\PYZsh{}df.ratio = df.height / df.weight}

\PY{n}{df}\PY{o}{.}\PY{n}{head}\PY{p}{(}\PY{p}{)}
\end{Verbatim}
\end{tcolorbox}

            \begin{tcolorbox}[breakable, size=fbox, boxrule=.5pt, pad at break*=1mm, opacityfill=0]
\prompt{Out}{outcolor}{25}{\boxspacing}
\begin{Verbatim}[commandchars=\\\{\}]
   height  weight  count     ratio
0     149      76      1  1.960526
1     161      56      1  2.875000
2     168      63      1  2.666667
3     160      71      1  2.253521
4     146      72      1  2.027778
\end{Verbatim}
\end{tcolorbox}
        
    Pojawiła się nowa kolumna o nazwie \texttt{ratio}. Zwróć uwagę, że
dzieląc jedną kolumnę przez drugą, traktowaliśmy to jako pojedynczą
wartość (ale pod spodem są wektory). Wiem, że część osób, które
przyzwyczaiły się do standardowych \texttt{for} (czyli zwykłej pętli)
mają trudności z przełączeniem myślenia na sposób wektorowy.

Jeśli również tak masz, to zatrzymaj się na chwilę i spróbuj się
przestawić. Zamiast operować pojedynczymi wartościami, operujesz
wektorami:
\texttt{df{[}\textquotesingle{}height\textquotesingle{}{]}\ /\ df{[}\textquotesingle{}weight\textquotesingle{}{]}}
, gdzie \texttt{height} jest wektorem (lub \texttt{Series} w
terminologii pandas) i \texttt{weight} również jest wektorem, a wynik
działania również jest wektorem :).

\hypertarget{filtrowanie}{%
\subsection{Filtrowanie}\label{filtrowanie}}

Kolejna operacja, która jest bardzo potrzebna, to filtrowanie. Załóżmy,
że chcemy zostawić w tabelce ludzi, którzy mają mniej niż 145 cm lub
więcej niż 195 cm.

    \begin{tcolorbox}[breakable, size=fbox, boxrule=1pt, pad at break*=1mm,colback=cellbackground, colframe=cellborder]
\prompt{In}{incolor}{26}{\boxspacing}
\begin{Verbatim}[commandchars=\\\{\}]
\PY{n}{df}\PY{p}{[} \PY{p}{(}\PY{n}{df}\PY{o}{.}\PY{n}{height} \PY{o}{\PYZlt{}} \PY{l+m+mi}{145}\PY{p}{)} \PY{o}{|} \PY{p}{(}\PY{n}{df}\PY{o}{.}\PY{n}{height} \PY{o}{\PYZgt{}} \PY{l+m+mi}{195}\PY{p}{)} \PY{p}{]}
\end{Verbatim}
\end{tcolorbox}

            \begin{tcolorbox}[breakable, size=fbox, boxrule=.5pt, pad at break*=1mm, opacityfill=0]
\prompt{Out}{outcolor}{26}{\boxspacing}
\begin{Verbatim}[commandchars=\\\{\}]
    height  weight  count     ratio
9      200      57      1  3.508772
14     140      77      1  1.818182
\end{Verbatim}
\end{tcolorbox}
        
    Zwróć uwagę na składnię.

Po pierwsze, jeśli jest więcej niż jeden filtr, należy pojedyncze
wyrażenia trzymać w nawiasach (to jest konieczne).

Po drugie pomiędzy wyrażeniami logicznymi jest operator, w tym przypadku
``lub'' czyli pionowa kreska (ang. pipe) ``\textbar{}''. Również może
być operator ``i'', wtedy należy użyć
\href{https://bit.ly/3rx5brO}{ampersandy/etki} ``\&''.

Po trzecie, wyrażeń logicznych może być więcej niż jedno, ale powyżej
trzech trudno jest to czytać i łatwo popełnić błąd.

    \hypertarget{usuwanie-kolumn}{%
\subsection{Usuwanie kolumn}\label{usuwanie-kolumn}}

Czasem chcemy usunąć kolumnę i możemy zrobić to na wiele sposobów,
jednak najlepszy jest jeden - użyć słowa kluczowego
\textbf{\texttt{del}}.

Jeśli natomiast odpalisz komórkę z usuwaniem dwa lub więcej razy, to
pojawi się wyjątek \texttt{KeyError}. Dzieje się tak dlatego, że kolumna
została już usunięta i trudno jest usunąć ją jeszcze raz (bo nie
istnieje).

Dlatego warto dodawać trochę więcej kodu, który najpierw sprawdzi, czy
kolumna istnieje i jeśli tak, to usuwa ją. Dlaczego tak? Dlatego, że
chcemy odpalać komórkę wiele razy i spodziewać się, że wynik za każdym
razem będzie ten sam. Jeśli tego jeszcze teraz nie rozumiesz, to uwierz
mi, że chcesz, aby taki był wynik :).

    \begin{tcolorbox}[breakable, size=fbox, boxrule=1pt, pad at break*=1mm,colback=cellbackground, colframe=cellborder]
\prompt{In}{incolor}{27}{\boxspacing}
\begin{Verbatim}[commandchars=\\\{\}]
\PY{k}{if} \PY{l+s+s1}{\PYZsq{}}\PY{l+s+s1}{ratio}\PY{l+s+s1}{\PYZsq{}} \PY{o+ow}{in} \PY{n}{df}\PY{p}{:} \PY{k}{del} \PY{n}{df}\PY{p}{[}\PY{l+s+s1}{\PYZsq{}}\PY{l+s+s1}{ratio}\PY{l+s+s1}{\PYZsq{}}\PY{p}{]}
\end{Verbatim}
\end{tcolorbox}

    \hypertarget{zadanie-1.2.4}{%
\section{Zadanie 1.2.4}\label{zadanie-1.2.4}}

Zrób \texttt{dataframe}, który na początek zawiera dwie kolumny: -
\textbf{age} (losowe wartości od 15 do 50) - \textbf{weight} (losowe
wartości od 50 do 80)

    \begin{tcolorbox}[breakable, size=fbox, boxrule=1pt, pad at break*=1mm,colback=cellbackground, colframe=cellborder]
\prompt{In}{incolor}{28}{\boxspacing}
\begin{Verbatim}[commandchars=\\\{\}]
\PY{k+kn}{import} \PY{n+nn}{pandas} \PY{k}{as} \PY{n+nn}{pd}

\PY{n}{np}\PY{o}{.}\PY{n}{random}\PY{o}{.}\PY{n}{seed}\PY{p}{(}\PY{l+m+mi}{1}\PY{p}{)}
\PY{n}{rows} \PY{o}{=} \PY{l+m+mi}{15}

\PY{n}{df} \PY{o}{=} \PY{n}{pd}\PY{o}{.}\PY{n}{DataFrame}\PY{p}{(}\PY{p}{\PYZob{}}
    \PY{l+s+s1}{\PYZsq{}}\PY{l+s+s1}{age}\PY{l+s+s1}{\PYZsq{}}\PY{p}{:} \PY{n}{np}\PY{o}{.}\PY{n}{random}\PY{o}{.}\PY{n}{randint}\PY{p}{(}\PY{l+m+mi}{15}\PY{p}{,} \PY{l+m+mi}{50}\PY{p}{,} \PY{n}{rows}\PY{p}{)}\PY{p}{,}
    \PY{l+s+s1}{\PYZsq{}}\PY{l+s+s1}{weight}\PY{l+s+s1}{\PYZsq{}}\PY{p}{:} \PY{n}{np}\PY{o}{.}\PY{n}{random}\PY{o}{.}\PY{n}{randint}\PY{p}{(}\PY{l+m+mi}{50}\PY{p}{,} \PY{l+m+mi}{80}\PY{p}{,} \PY{n}{rows}\PY{p}{)}\PY{p}{,}
\PY{p}{\PYZcb{}}\PY{p}{)}

\PY{n}{df}
\end{Verbatim}
\end{tcolorbox}

            \begin{tcolorbox}[breakable, size=fbox, boxrule=.5pt, pad at break*=1mm, opacityfill=0]
\prompt{Out}{outcolor}{28}{\boxspacing}
\begin{Verbatim}[commandchars=\\\{\}]
    age  weight
0    27      70
1    23      61
2    24      78
3    26      60
4    20      78
5    30      79
6    15      64
7    31      68
8    16      54
9    27      73
10   22      73
11   21      59
12   40      67
13   35      73
14   33      50
\end{Verbatim}
\end{tcolorbox}
        
    \#

👉 Kliknij tutaj (1 klik), aby zobaczyć podpowiedź 👈

Zainspiruj się powyższym przykładem i zmień nazwę kolumny na
\texttt{age}.

👉 Kliknij tutaj (1 klik), aby zobaczyć odpowiedź 👈

\begin{Shaded}
\begin{Highlighting}[]
\NormalTok{np.random.seed(}\DecValTok{0}\NormalTok{)}
\NormalTok{rows }\OperatorTok{=} \DecValTok{15}

\NormalTok{df }\OperatorTok{=}\NormalTok{ pd.DataFrame(\{}
    \StringTok{\textquotesingle{}age\textquotesingle{}}\NormalTok{: np.random.randint(}\DecValTok{15}\NormalTok{, }\DecValTok{50}\NormalTok{, rows),}
    \StringTok{\textquotesingle{}weight\textquotesingle{}}\NormalTok{: np.random.randint(}\DecValTok{50}\NormalTok{, }\DecValTok{80}\NormalTok{, rows),}
\NormalTok{\})}

\NormalTok{df}
\end{Highlighting}
\end{Shaded}

    Dawniej (teraz już są inne standardy, ale pomijam ten wątek) uważało
się, że dla chłopców/mężczyzn wzrost w cm to
\texttt{wzrost\ =\ waga\ +\ 110}.

Twoim zadaniem jest stworzyć nową kolumnę o nazwie \texttt{height},
która będzie równa \texttt{weight\ +\ 110}.

    \begin{tcolorbox}[breakable, size=fbox, boxrule=1pt, pad at break*=1mm,colback=cellbackground, colframe=cellborder]
\prompt{In}{incolor}{30}{\boxspacing}
\begin{Verbatim}[commandchars=\\\{\}]
\PY{n}{df}\PY{p}{[}\PY{l+s+s1}{\PYZsq{}}\PY{l+s+s1}{height}\PY{l+s+s1}{\PYZsq{}}\PY{p}{]} \PY{o}{=} \PY{n}{df}\PY{p}{[}\PY{l+s+s1}{\PYZsq{}}\PY{l+s+s1}{weight}\PY{l+s+s1}{\PYZsq{}}\PY{p}{]} \PY{o}{+} \PY{l+m+mi}{110}
\PY{n}{df}
\end{Verbatim}
\end{tcolorbox}

            \begin{tcolorbox}[breakable, size=fbox, boxrule=.5pt, pad at break*=1mm, opacityfill=0]
\prompt{Out}{outcolor}{30}{\boxspacing}
\begin{Verbatim}[commandchars=\\\{\}]
    age  weight  height
0    27      70     180
1    23      61     171
2    24      78     188
3    26      60     170
4    20      78     188
5    30      79     189
6    15      64     174
7    31      68     178
8    16      54     164
9    27      73     183
10   22      73     183
11   21      59     169
12   40      67     177
13   35      73     183
14   33      50     160
\end{Verbatim}
\end{tcolorbox}
        
    \hypertarget{przydatne-linki}{%
\subsection{Przydatne linki:}\label{przydatne-linki}}

\begin{itemize}
\tightlist
\item
  Ksiażka: \href{https://bit.ly/31ubHFw}{Python Data Science Handbook}
\item
  \href{https://bit.ly/3cwans0}{Webinar - python/numpy} +
  \href{https://bit.ly/3u58eJQ}{github}
\item
  \href{https://bit.ly/3dcQd5r}{Webinar - pandas} +
  \href{https://bit.ly/3u2jMxb}{github}
\end{itemize}

    \begin{tcolorbox}[breakable, size=fbox, boxrule=1pt, pad at break*=1mm,colback=cellbackground, colframe=cellborder]
\prompt{In}{incolor}{ }{\boxspacing}
\begin{Verbatim}[commandchars=\\\{\}]

\end{Verbatim}
\end{tcolorbox}


    % Add a bibliography block to the postdoc
    
    
    
\end{document}
